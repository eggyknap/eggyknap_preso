% Fun with Transactions
\documentclass{beamer}
\usetheme{Ilmenau}
%\usepackage{tipa}
\usepackage{color}
\usepackage{listings}
\usepackage[utf8,latin9]{inputenc}
\usepackage[T1]{fontenc}
%\usepackage{babel}
\beamertemplatenavigationsymbolsempty

\begin{document}
\title{Fun with Transactions}
% subtitle: what robust transactions can do for you, and how they can do it for free
\author{Joshua Tolley -- eggyknap -- End Point Corporation}

% outline:
% introductory blabber
% What's a transaction?
% - Classic banking example
% - More commonplace example (i.e. everyone knows financials need to be robust; what about other stuff)
% How do you implement such things
% - WAL log
% - Locking & MVCC (explain vacuum)
% - Explain clog, xmin, xmax? Too much detail?
% Visibility
% - dirty, phantom, and non-repeatable reads. Isolation levels
% Go back to examples for a while to emphasize why it's important to have transactions
% - You need transactions whether you're ready to admit it or not
% - The application has units of work it needs to keep atomic. What's more, only the application knows the boundaries of those units
% - Unfortunately often your ORM disagrees with these last two
% - Lots of programmers are too lazy to catch and handle transaction errors (e.g. locking failures, serialization failures)
% - Some people understand transactions' importance enough to transaction-enable other stuff. Message queues, data transformation services
% Multiple transactional systems can coordinate via 2PC
% - Java has the XA spec, and transaction managers. Federation software is also a possibility
% - There aren't many non-Java transaction managers
% - You can have one (but not more) non-2PC capable devices in a commit group, if it's done right
% Problems with pathological behavior
% - Locking tons of stuff
% - Long transactions are bad, and not just because of PostgreSQL and vacuum
% - Serialization failures
% What to do about it all?... 

\end{document}
