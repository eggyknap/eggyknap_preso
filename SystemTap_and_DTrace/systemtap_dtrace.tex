% Probing PostgreSQL with SystemTap and DTrace
\documentclass{beamer}
\usetheme{Berlin}
%\usepackage{tipa}
\usepackage{color}
\usepackage{listings}
\usepackage[utf8,latin9]{inputenc}
\usepackage[T1]{fontenc}
%\usepackage{babel}
\beamertemplatenavigationsymbolsempty

\begin{document}
\title{Probing PostgreSQL with SystemTap and DTrace}
\author{Joshua Tolley -- eggyknap -- End Point Corporation
}

% outline:
% Purpose
%  Sometimes logging isn't enough, and adding extra logging code is certainly not always possible or easy
% Start with DTrace
%  dtrace virtual machine in kernel running bytecode vs. stap compiled to C -> kernel module
%  Mark-based tracing. See -l option
% Show some examples
%  Introduce functions, variables, aggregates, etc. in D
% Show the same thing in systemtap
%  How it works: requires utrace patches for systemtap to do userland. Freebsd doesn't support userland tracing yet
%  Show off mark-based probes
%  Now show off debug-based probes
% Samples / Language introduction
%  sample - Show off some taps / probes while doing this
%  sample - Where am I getting IO, perhaps? What's in cache and what's not?
%  language - Does systemtap have thread-local variables like dtrace does?
% Some advanced stuff
%  Aggregations
% Other usage bits (what tracepoints are there, for instance)

\frame{\titlepage}
%\frame{\tableofcontents}

%\section{The setting}
\begin{frame}
    Sometimes error messages are inadequate
    \begin{figure}[b]
    \begin{centering}
    \scalebox{.75}{\includegraphics{arguments.png}}
    \end{centering}
    \end{figure}
\end{frame}

\begin{frame}
    Sometimes error logs aren't much better
    \begin{figure}[b]
    \begin{centering}
    \scalebox{.6}{\includegraphics{long_logging_clipped.png}}
    \end{centering}
    \end{figure}
\end{frame}

\begin{frame}
    Often there exists a tool to get the information you're after, but...
    \begin{itemize}
        \item<2->You've never heard of it
        \item<3->No one else has ever heard of it
        \item<4->Even when people *have* heard of it, its options, usage, and output differ completely from the other tools you've used to get to this point, so integration is painful
        \item<5->The more specific the tool, the more output you need to sort through
    \end{itemize}
\end{frame}

%\section{Our hero appears...}
\begin{frame}
    Enter Dynamic Tracing...
\end{frame}

\begin{frame}
    Dynamic tracing allows users to instrument production systems in (ideally) whatever ways they want, without (ideally) breaking things in the process
\end{frame}

%\section{Major players}
\begin{frame}
    Important packages
    \begin{itemize}
        \item<2-> DTrace
        \begin{itemize}
            \item Available on [Open]Solaris, FreeBSD, and OSX
            \item FreeBSD's version is less mature than Solaris, and thus has significant limitations
            \item Who runs OSX servers, anyway?
            \item Any guesses how much life OpenSolaris has left?
        \end{itemize}
        \item<3->SystemTap
        \begin{itemize}
            \item Linux only
            \item Less mature than DTrace
            \item Requires kernel tracing patches and/or debug information to be really useful
        \end{itemize}
        \item<4-> There are others (ProbeVue on AIX, other Linux packages with the same goal)
        \begin{itemize}
            \item ...but I won't be talking about them (or I would have put them in the talk's title, now, wouldn't I?)
        \end{itemize}
    \end{itemize}
\end{frame}

\begin{frame}
    The basic idea:
    \begin{itemize}
        \item Software (kernel, libraries, PostgreSQL, etc.) is equipped with various probe points
        \item Simple, C-like scripting language describes which probes are used and what to do with them
        \item Script allows variables, screen output, control structures, etc.
        \item Compiled and run at the kernel level
        \item Runtime environment includes security and fault tolerance protections
        \item Ideally monitoring overhead is minimal for things that are probed, and nothing for unused probes
    \end{itemize}
\end{frame}

\begin{frame}
    An example...
\end{frame}

\begin{frame}[fragile]
    "Hello, World!", DTrace style
    \begin{example}
    \begin{lstlisting}
BEGIN 
{ 
 trace("Hello, World!"); 
 exit(0); 
}
    \end{lstlisting}
    \end{example}
    \begin{example}
    \begin{verbatim}
dtrace -s Hello.d 
dtrace: script 'Hello.d' matched 1 probe
CPU   ID   FUNCTION:NAME
  0    1      :BEGIN   Hello, World!
    \end{verbatim}
    \end{example}
\end{frame}

\begin{frame}
    A more complex example
    \begin{example}
    \begin{lstlisting}
    test
    \end{lstlisting}
    \end{example}
\end{frame}

%syscall::write:entry
%/pid == $1/
%{
%    self->buf = arg1;
%}
%
%syscall::write:return
%/pid == $1/
%{
%    self->text = copyin(self->buf, arg0);
%    printf("%s", stringof(self->text));
%}
%\section{How does it work?}
%
%\begin{frame}
%"The degree of normality in a database is inversely proportional to that of its DBA."
%- Anon, twitter
%\end{frame}
%
%\section{Why and Why Not}
%\begin{frame}
%    \frametitle{Why Not Do Stuff in SQL}
%    \begin{itemize}
%        \item Databases are harder to replicate, if you \textbf{really} need to scale out
%        \pause
%        \begin{itemize}
%            \item Often, one complex SQL query is more efficient than several simple ones
%            \pause
%            \item Sometimes, indeed, it's useful to reduce the load on the database by moving logic into the application. Be careful doing this
%            \pause
%            \begin{itemize}
%                \item c.f. Premature Optimization
%                \pause
%            \end{itemize}
%        \end{itemize}
%        \item More complex queries are harder to write and debug
%        \pause
%        \begin{itemize}
%            \item True. But so is more complex programming.
%            \pause
%        \end{itemize}
%        \item More complex queries are harder for the next guy to maintain
%        \pause
%        \item Also, good DBAs are often more expensive than good programmers
%        \pause
%        \begin{itemize}
%            \item These are both true. But complex programming is also hard for the next guy to maintain
%            \pause
%            \item Of all the reasons not to write fluent SQL, this is probably the most widely applicable
%        \end{itemize}
%    \end{itemize}
%\end{frame}
%
%
%\begin{frame}
%    \frametitle{Why do stuff in SQL?}
%    \begin{itemize}
%        \item The database is more efficient than your application for processing big chunks of data
%        \pause
%        \begin{itemize}
%            \item ...especially if your code is in an interpreted language
%            \pause
%        \end{itemize}
%        \item The database is better tested than your application
%        \pause
%        \begin{itemize}
%            \item Applications trying to do what SQL should be doing often get big and complex quickly
%            \pause
%            \item ...and also buggy quickly 
%            \pause
%        \end{itemize}
%        \item That's what the database is there for
%        \pause
%        \item SQL is designed to express relations and conditions on them. Your application's language isn't.
%        \pause
%        \item A better understanding of SQL allows you to write queries that perform better
%    \end{itemize}
%\end{frame}
%
%\begin{frame}
%    \frametitle{Why do stuff in SQL?}
%    In short, the database exists to manage data, and your application exists to handle business logic. Write software accordingly.
%\end{frame}
%
%\begin{frame}
%    So let's get started...
%\end{frame}
%
%\begin{frame}[fragile]
%    \frametitle{Tables we'll use}
%
%   \begin{columns}[l]
%       \column{1.5in}
%       \begin{verbatim}
%# SELECT * FROM a;
% id | value
%----+-------
%  1 | a1
%  2 | a2
%  3 | a3
%  4 | a4
%(4 rows)
%       \end{verbatim}
%       \column{1.5in}
%       \begin{verbatim}
%# SELECT * FROM b;
% id | value
%----+-------
%  5 | b5
%  4 | b4
%  3 | b3
%  6 | b6
%(4 rows)
%       \end{verbatim}
%   \end{columns}
%\end{frame}
%
%\section{Joins}
%
%\begin{frame}
%    \frametitle{JOINs}
%    \begin{itemize}
%        \item If you want data from multiple tables, you probably want a join
%        \pause
%        \begin{itemize}
%            \item ...but see also Subqueries, later on
%            \pause
%        \end{itemize}
%        \item There are several different kinds of joins
%    \end{itemize}
%\end{frame}
%
%\begin{frame}[fragile]
%    \frametitle{JOINs}
%    \begin{verbatim}
%<table1> [alias1]
%    [ [ [NATURAL] [ [FULL | RIGHT | LEFT] [OUTER] |
%    INNER] ] | CROSS ] JOIN
%<table2> [alias2]
%    [USING (...) |
%    ON (<value1> <op> <value2>
%        [,<value3> <op> <value4>...] ) ]
%    \end{verbatim}
%\end{frame}
%
%\subsection{CROSS JOIN}
%
%\begin{frame}
%    \frametitle{CROSS JOIN}
%    \begin{itemize}
%        \item SELECT \textless ... \textgreater FROM table1 JOIN table2
%        \pause
%        \item With no explicit join type and no join qualifiers (an ON clause, WHERE clause involving both relations, etc.) this is a CROSS JOIN
%        \pause
%        \item Equivalent to
%        \begin{itemize}
%            \item SELECT \textless ... \textgreater FROM table1, table2
%            \pause
%            \item SELECT \textless ... \textgreater FROM table1 CROSS JOIN table2
%            \pause
%        \end{itemize}
%        \item "Cartesian product" of the two relations
%        \pause
%        \begin{itemize}
%            \item Combines every row of table1 with every row of table2
%            \pause
%            \item Makes \textbf{LOTS} of rows, and can thus be very slow
%        \end{itemize}
%    \end{itemize}
%\end{frame}
%
%\begin{frame}[fragile]
%    \frametitle{CROSS JOIN}
%    \begin{verbatim}
%# SELECT * FROM a, b;
% id | value | id | value
%----+-------+----+-------
%  1 | a1    |  5 | b5
%  1 | a1    |  4 | b4
%<snip>
%  3 | a3    |  3 | b3
%  3 | a3    |  6 | b6
%  4 | a4    |  5 | b5
%  4 | a4    |  4 | b4
%  4 | a4    |  3 | b3
%  4 | a4    |  6 | b6
%(16 rows)
%    \end{verbatim}
%\end{frame}
%
%\subsection{INNER JOIN}
%\begin{frame}
%    \frametitle{INNER JOIN}
%    \begin{itemize}
%        \item SELECT \textless ...\textgreater FROM table1 INNER JOIN table2 ON (table1.field = table2.field ...)
%        \pause
%        \item Only returns rows satisfying the ON condition
%        \pause
%        \item Equivalent to a CROSS JOIN with a WHERE clause
%    \end{itemize}
%\end{frame}
%
%\begin{frame}[fragile]
%    \frametitle{INNER JOIN}
%    \begin{verbatim}
%# SELECT * FROM a INNER JOIN b USING (id);
% id | value | value
%----+-------+-------
%  3 | a3    | b3
%  4 | a4    | b4
%(2 rows)
%    \end{verbatim}
%\end{frame}
%
%\subsection{OUTER JOIN}
%\begin{frame}
%    \frametitle{OUTER JOIN}
%    \begin{itemize}
%        \item Return all rows from one or both relations
%        \pause
%        \item LEFT: Return all rows from the relation on the left
%        \pause
%        \item RIGHT: Return all rows from the relation on the right
%        \pause
%        \item FULL: Return all rows from both relations
%        \pause
%        \item Returns nulls for values from one relation when it contains to match with the other relation
%        \pause
%        \item The OUTER keyword is redundant
%        \pause
%        \item Requires ON or USING clause
%    \end{itemize}
%\end{frame}
%
%\begin{frame}[fragile]
%    \frametitle{LEFT JOIN}
%    \begin{verbatim}
%# SELECT * FROM a LEFT JOIN b USING (id);
% id | value | value
%----+-------+-------
%  1 | a1    |
%  2 | a2    |
%  3 | a3    | b3
%  4 | a4    | b4
%(4 rows)
%    \end{verbatim}
%\end{frame}
%
%\begin{frame}[fragile]
%    \frametitle{RIGHT JOIN}
%    \begin{verbatim}
%# SELECT * FROM a RIGHT JOIN b USING (id);
% id | value | value
%----+-------+-------
%  3 | a3    | b3
%  4 | a4    | b4
%  5 |       | b5
%  6 |       | b6
%(4 rows)
%    \end{verbatim}
%\end{frame}
%
%\begin{frame}[fragile]
%    \frametitle{FULL JOIN}
%    \begin{verbatim}
%# select * from a full join b using (id);
% id | value | value
%----+-------+-------
%  1 | a1    |
%  2 | a2    |
%  3 | a3    | b3
%  4 | a4    | b4
%  5 |       | b5
%  6 |       | b6
%(6 rows)
%    \end{verbatim}
%\end{frame}
%
%\begin{frame}[fragile]
%    \frametitle{Applications}
%    Find rows with no match in table b: \\
%    \begin{verbatim}
%# SELECT * FROM a LEFT JOIN b USING (id)
%    WHERE b.value IS NULL;
% id | value | value
%----+-------+-------
%  1 | a1    |
%  2 | a2    |
%(2 rows)
%    \end{verbatim}
%\end{frame}
%
%\subsection{NATURAL JOIN}
%\begin{frame}
%    \frametitle{NATURAL JOIN}
%    \begin{itemize}
%        \item NATURAL is syntactic sugar to match all columns with the same name
%    \end{itemize}
%\end{frame}
%
%\begin{frame}[fragile]
%    \frametitle{NATURAL JOIN}
%    \begin{verbatim}
%# SELECT * FROM a NATURAL FULL JOIN b;
% id | value
%----+-------
%  1 | a1
%  2 | a2
%  3 | a3
%  3 | b3
%  4 | a4
%  4 | b4
%  5 | b5
%  6 | b6
%(8 rows)
%    \end{verbatim}
%%    \vspace{10pt}
%    This looked for matches in both the \emph{id} and \emph{value} columns, so no rows matched. It returned all rows of both relations because it's a FULL JOIN.
%\end{frame}
%
%\subsection{Self Joins}
%\begin{frame}
%    \frametitle{Self Joins}
%    \begin{itemize}
%        \item "Self joins" are particularly counterintuitive
%        \pause
%        \item Joins one table to itself
%        \pause
%        \item It helps to give the table two different aliases
%    \end{itemize}
%\end{frame}
%
%\begin{frame}[fragile]
%    \frametitle{Self Joins}
%    Find all employees' names, and each employee's manager
%    %\vspace{10pt}
%    \begin{verbatim}
%SELECT
%    e.first || ' ' || e.last,
%    (SELECT
%        m.first || ' ' || m.last
%     FROM employee m
%     WHERE m.id = e.manager);
%    \end{verbatim}
%    %\vspace{10pt}
%    ... will generally be much faster rewritten as ...
%    %\vspace{10pt}
%    \begin{verbatim}
%SELECT
%    e.first || ' ' || e.last,
%    m.first || ' ' || m.last
%FROM
%    employee e
%    JOIN employee m ON (e.manager = m.id)
%    \end{verbatim}
%    %\vspace{10pt}
%\end{frame}
%
%\section{Other Useful Operations}
%\begin{frame}
%    More useful operations...
%\end{frame}
%
%\subsection{Subqueries}
%\begin{frame}
%    \frametitle{Subqueries}
%    \begin{itemize}
%        \item Embeds one query within another
%        \pause
%        \item Examples (some bad, some good)
%        \pause
%        \begin{itemize}
%            \item SELECT id FROM table WHERE field = (SELECT MAX(field) FROM table)
%            \pause
%            \item SELECT id, (SELECT COUNT(*) FROM table2 WHERE id = table1.id) FROM table1
%            \pause
%            \item SELECT a, b FROM (SELECT a, COUNT(*) AS c FROM table1) t1 JOIN (SELECT b, COUNT(*) AS c FROM table2) t2 on (t1.c = t2.c)
%            \pause
%            \begin{itemize}
%                \item You can join subqueries just like you'd join tables
%            \end{itemize}
%        \end{itemize}
%    \end{itemize}
%\end{frame}
%
%\subsection{Set Operations}
%\begin{frame}
%    \frametitle{Set Operations}
%    \begin{itemize}
%        \item INTERSECT
%        \pause
%        \begin{itemize}
%            \item Returns the intersection of two sets
%            \pause
%            \item Doesn't exist in MySQL
%            \pause
%            \item SELECT (SELECT a, b FROM table1) INTERSECT (SELECT c, d FROM table2)
%            \pause
%        \end{itemize}
%        \item UNION
%        \pause
%        \begin{itemize}
%            \item Appends one set of rows to another set with matching column types
%            \pause
%            \item SELECT a FROM table1 UNION SELECT b FROM table2
%            \pause
%        \end{itemize}
%        \item EXCEPT
%        \pause
%        \begin{itemize}
%            \item Returns rows in one SELECT that aren't in another SELECT
%            \pause
%            \item SELECT a FROM table1 EXCEPT SELECT b FROM table2
%        \end{itemize}
%    \end{itemize}
%\end{frame}
%
%\subsection{Common Operations}
%\begin{frame}
%    \frametitle{Common Operations}
%    \begin{itemize}
%        \item COALESCE(a, b)
%        \pause
%        \begin{itemize}
%            \item If \texttt{a} is null, return \texttt{b}, else return \texttt{a}
%            \pause
%            \item SELECT COALESCE(first, '\textless NULL\textgreater ') FROM table
%            \pause
%            \item Oracle calls this NVL()
%            \pause
%        \end{itemize}
%        \item CASE...WHEN
%        \pause
%        \begin{itemize}
%            \item Conditional operation
%            \pause
%            \item SELECT CASE WHEN langused IN ('Lisp', 'OCaml', 'Haskell') THEN 'Functional' ELSE 'Imperative' AS langtype FROM software
%        \end{itemize}
%    \end{itemize}
%\end{frame}
%
%\begin{frame}[fragile]
%    \frametitle{Series Generation}
%    \begin{itemize}
%        \item generate\_series() in PostgreSQL; might be something else in other databases
%        \pause
%        \item Returns a series of numbers
%        \pause
%        \item Can be used like a \texttt{for} loop (example given later)
%    \end{itemize}
%    \begin{verbatim}
%# SELECT * FROM generate_series(1, 5);
% generate_series
%-----------------
%               1
%               2
%               3
%               4
%               5
%(5 rows)
%    \end{verbatim}
%\end{frame}
%
%\section{Advanced Operations}
%\subsection{Common Table Expressions}
%\begin{frame}
%    \frametitle{Common Table Expressions}
%    \begin{itemize}
%        \item Abbreviated CTEs
%        \pause
%        \item Fairly advanced; not available in all databases
%        \pause
%        \begin{itemize}
%            \item Not in PostgreSQL before v. 8.4, or any version of MySQL
%        \end{itemize}
%        \item It's just like defining a one-time view for your query
%        \pause
%        \item One major benefit: CTEs allow recursion
%        \pause
%        \begin{itemize}
%            \item Recursing with CTEs is much more efficent than processing recursive data in your application
%        \end{itemize}
%    \end{itemize}
%\end{frame}
%
%\begin{frame}[fragile]
%    \frametitle{A Simple CTE Example}
%    \begin{verbatim}
%# SELECT * FROM GENERATE_SERIES(1,3)
%CROSS JOIN
%    (SELECT * FROM GENERATE_SERIES(8,9)) AS f;
% generate_series | generate_series
%-----------------+-----------------
%               1 |               8
%               1 |               9
%               2 |               8
%               2 |               9
%               3 |               8
%               3 |               9
%(6 rows)
%    \end{verbatim}
%\end{frame}
%
%\begin{frame}[fragile]
%    \frametitle{A Simple CTE Example}
%    \begin{verbatim}
%# WITH t AS (
%    SELECT * FROM GENERATE_SERIES(8,9)
%)
%SELECT * FROM GENERATE_SERIES(1,3)
%CROSS JOIN t;
% generate_series | generate_series
%-----------------+-----------------
%               1 |               8
%               1 |               9
%               2 |               8
%               2 |               9
%               3 |               8
%               3 |               9
%(6 rows)
%    \end{verbatim}
%\end{frame}
%
%\begin{frame}
%    \begin{center}
%    That last example was a bit cheesy, but the technique can be useful for complex queries in several parts
%    \end{center}
%\end{frame}
%
%\begin{frame}[fragile]
%    \frametitle{Recursion}
%    Start with this:
%    \begin{verbatim}
%# SELECT * FROM employee;
% first  |   last   | id | manager
%--------+----------+----+---------
% john   | doe      |  1 |
% fred   | rogers   |  2 |       1
% speedy | gonzales |  3 |       1
% carly  | fiorina  |  4 |       1
% hans   | reiser   |  5 |       2
% johnny | carson   |  6 |       5
% martha | stewart  |  7 |       3
%(7 rows)
%    \end{verbatim}
%\end{frame}
%
%\begin{frame}[fragile]
%    \frametitle{Recursion}
%    Recursive CTE to retrieve management hierarchy:
%    \begin{verbatim}
%# WITH RECURSIVE t (id, managernames) AS (
%    SELECT e.id, first || ' ' || last
%        AS managernames
%    FROM employee e WHERE manager IS NULL
%        UNION ALL
%    SELECT e.id,
%    first || ' ' || last || ', ' || managernames
%        AS managernames
%    FROM employee e
%    JOIN t ON (e.manager = t.id)
%    WHERE manager IS NOT NULL
%)
%SELECT e.id, first || ' ' || last AS name,
%    managernames
%FROM employee e JOIN t ON (e.id = t.id);
%    \end{verbatim}
%\end{frame}
%
%\begin{frame}[fragile]
%    \frametitle{Recursion}
%    ...and get this...
%    \begin{verbatim}
% id |     name        |  managernames                   
%----+-----------------+-----------------------------
%  1 | john doe        | john doe
%  2 | fred rogers     | fred rogers, john doe
%  3 | speedy gonzales | speedy gonzales, john doe
%  4 | carly fiorina   | carly fiorina, john doe
%  5 | hans reiser     | hans reiser, fred rogers,
%    |                 |  john doe
%  6 | johnny carson   | johnny carson, hans reiser,
%    |                 |  fred rogers, john doe
%  7 | martha stewart  | martha stewart, speedy
%    |                 |  gonzales, john doe
%(7 rows)
%    \end{verbatim}
%\end{frame}
%
%\begin{frame}[fragile]
%    \frametitle{Fractals in SQL}
%    \tiny
%    \begin{verbatim}
%WITH RECURSIVE x(i) AS
%(VALUES(0) UNION ALL SELECT i + 1 FROM x WHERE i < 101),
%Z(Ix, Iy, Cx, Cy, X, Y, I)
%AS (
%    SELECT Ix, Iy, X::float, Y::float, X::float, Y::float, 0
%    FROM (SELECT -2.2 + 0.031 * i, i FROM x) AS xgen(x,ix)
%        CROSS JOIN
%    (SELECT -1.5 + 0.031 * i, i FROM x) AS ygen(y,iy)
%        UNION ALL
%    SELECT
%        Ix, Iy, Cx, Cy, X * X - Y * Y + Cx AS X,
%        Y * X * 2 + Cy, I + 1
%    FROM Z
%    WHERE X * X + Y * Y < 16.0 AND I < 27),
%Zt (Ix, Iy, I) AS (
%    SELECT Ix, Iy, MAX(I) AS I
%    FROM Z GROUP BY Iy, Ix
%    ORDER BY Iy, Ix
%)
%SELECT array_to_string(
%    array_agg(
%        SUBSTRING(' .,,,-----++++%%%%@@@@#### ',
%            GREATEST(I,1), 1)
%    ),''
%)
%FROM Zt GROUP BY Iy ORDER BY Iy;
%    \end{verbatim}
%    \normalsize
%    (yes, this query is SQL-spec compliant)
%\end{frame}
%
%\begin{frame}
%    \scalebox{.3}{\includegraphics{mandelbrot.png}}
%\end{frame}
%
%\subsection{Window Functions}
%\begin{frame}
%    \frametitle{Window Functions}
%    \begin{itemize}
%        \item Like CTEs, these are quite advanced
%        \pause
%        \item Also unavailable in MySQL, and PostgreSQL before 8.4
%        \pause
%        \item Allow ranking, moving averages
%        \pause
%        \item Like a set-returning aggregate function. Window functions return results for each row based on a "window" of related rows
%    \end{itemize}
%\end{frame}
%
%\begin{frame}[fragile]
%    \frametitle{Window Functions}
%    If our employee table had department and salary information...
%    \begin{verbatim}
%# SELECT first, last, salary, department
%    FROM employee;
% first  |   last   | salary |   department
%--------+----------+--------+----------------
% fred   | rogers   |  97000 | sales
% carly  | fiorina  |  95000 | sales
% johnny | carson   |  89000 | sales
% speedy | gonzales |  96000 | development
% hans   | reiser   |  93000 | development
% martha | stewart  |  90000 | development
% john   | doe      |  99000 | administration
%(7 rows)
%    \end{verbatim}
%\end{frame}
%
%\begin{frame}[fragile]
%    \frametitle{Window Functions Example}
%    Rank employees in each department by salary
%    \begin{verbatim}
%SELECT first, last, salary, department,
%    RANK() OVER (
%        PARTITION BY department
%        ORDER BY salary DESC
%    )
%FROM employee
%    \end{verbatim}
%\end{frame}
%
%\begin{frame}[fragile]
%    \frametitle{Window Functions Example}
%    ... and get this:
%    \footnotesize
%    \begin{verbatim}
% first  |   last   | salary |   department   | rank
%--------+----------+--------+----------------+------
% john   | doe      |  99000 | administration |    1
% speedy | gonzales |  96000 | development    |    1
% hans   | reiser   |  93000 | development    |    2
% martha | stewart  |  90000 | development    |    3
% fred   | rogers   |  97000 | sales          |    1
% carly  | fiorina  |  95000 | sales          |    2
% johnny | carson   |  89000 | sales          |    3
%(7 rows)
%    \end{verbatim}
%    \normalsize
%\end{frame}
%
%\section{Real, Live Queries}
%\subsection{Something Simple}
%\begin{frame}
%    \begin{center}
%       Real, live queries
%    \end{center}
%\end{frame}
%
%\begin{frame}[fragile]
%    \frametitle{Something Simple}
%    The slow version:
%    \begin{verbatim}
%SELECT DISTINCT(sync) FROM bucardo.bucardo_rate
%ORDER BY 1
%    \end{verbatim}
%    The fast version:
%    \begin{verbatim}
%SELECT name FROM sync WHERE EXISTS (
%    SELECT 1 FROM bucardo_rate
%    WHERE sync = name LIMIT 1)
%ORDER BY 1
%    \end{verbatim}
%\end{frame}
%
%\begin{frame}
%    \frametitle{Something Simple}
%    \begin{itemize}
%        \item The \emph{bucardo\_rate} table is huge, with few distinct values
%        \pause
%        \item finding "DISTINCT sync" requires a \textbf{\emph{long}} table scan
%        \pause
%        \item The \emph{sync} table contains a list of all possible values in the \emph{bucardo\_rate.sync} column
%        \pause
%        \item So instead of a big table scan, we scan the small table, and filter out values can't find in \emph{bucardo\_rate}
%    \end{itemize}
%\end{frame}
%
%\subsection{Something Fun}
%\begin{frame}
%    \begin{center}Something Fun\end{center}
%\end{frame}
%
%\begin{frame}[fragile]
%    \tiny
%    \begin{verbatim}
%SELECT
%    id, idname,
%    COALESCE(ROUND(AVG(synctime)::NUMERIC, 1), 0) AS avgtime,
%    COALESCE(SUM(total), 0) AS count
%FROM (
%    SELECT slavecommit,
%    EXTRACT(EPOCH FROM slavecommit - mastercommit) AS synctime,
%    total
%    FROM bucardo.bucardo_rate
%    WHERE sync = 'RO_everything' AND
%    mastercommit > (NOW() - (15 + 1) * INTERVAL '1 HOUR')
%) i
%RIGHT JOIN (
%    SELECT id, idname,
%        TO_TIMESTAMP(start - start::INTEGER % 3600) AS start,
%        TO_TIMESTAMP(stop - stop::INTEGER % 3600) AS stop
%    FROM (
%        SELECT id,
%            TO_CHAR(NOW() - id * INTERVAL '1 HOUR', 
%                'Dy Mon DD HH:MI AM') AS idname,
%            EXTRACT(EPOCH FROM NOW() - id * INTERVAL '1 HOUR') AS start,
%            EXTRACT(EPOCH FROM NOW() - (id - 1) * INTERVAL '1 HOUR') AS stop
%        FROM (
%            SELECT GENERATE_SERIES(1, 15) AS id
%        ) f
%    ) g
%) h ON (slavecommit BETWEEN start AND stop)
%GROUP BY id, idname
%ORDER BY id DESC;
%    \end{verbatim}
%\end{frame}
%
%\begin{frame}
%    \frametitle{Something Fun}
%    \begin{itemize}
%        \item The table contains replication data
%        \pause
%        \begin{itemize}
%            \item Time of commit on master
%            \pause
%            \item Time of commit on slave
%            \pause
%            \item Number of rows replicated
%            \pause
%        \end{itemize}
%        \item The user wants a graph of replication speed over time, given a user-determined range of time
%    \end{itemize}
%\end{frame}
%
%\begin{frame}[fragile]
%    \frametitle{Something Fun}
%    We want to average replication times over a series of buckets. The first
%    part of our query creates those buckets, based on
%    \texttt{generate\_series()}. Here we create buckets for 15 hours
%    \begin{verbatim}
%SELECT 
%    id,
%    TO_CHAR(NOW() - id * INTERVAL '1 HOUR',
%        'Dy Mon DD HH:MI AM') AS idname,
%    EXTRACT(EPOCH FROM NOW() - id *
%        INTERVAL '1 HOUR') AS start,
%    EXTRACT(EPOCH FROM NOW() - (id - 1) *
%        INTERVAL '1 HOUR') AS stop
%FROM (
%    SELECT GENERATE_SERIES(1, 15) AS id
%) f
%    \end{verbatim}
%\end{frame}
%
%\begin{frame}[fragile]
%    \frametitle{Something Fun}
%    This gives us:
%    \scriptsize
%    \begin{verbatim}
% id |       idname        |      start       |       stop
%----+---------------------+------------------+------------------
%  1 | Sat Mar 14 10:23 PM | 1237091036.95657 | 1237094636.95657
%  2 | Sat Mar 14 09:23 PM | 1237087436.95657 | 1237091036.95657
%  3 | Sat Mar 14 08:23 PM | 1237083836.95657 | 1237087436.95657
%  4 | Sat Mar 14 07:23 PM | 1237080236.95657 | 1237083836.95657
%...
%    \end{verbatim}
%    \normalsize
%\end{frame}
%
%\begin{frame}[fragile]
%    \frametitle{Something Fun}
%    Make the buckets end on nice time boundaries:
%    \begin{verbatim}
%SELECT id, idname,
%    TO_TIMESTAMP(start - start::INTEGER % 3600)
%        AS start,
%    TO_TIMESTAMP(stop - stop::INTEGER % 3600)
%        AS stop
%FROM (
%    -- The bucket query, shown earlier, goes here
%) g
%    \end{verbatim}
%\end{frame}
%
%\begin{frame}[fragile]
%    \frametitle{Something Fun}
%    That gives us this:
%    \tiny
%    \begin{verbatim}
% id |       idname        |             start             |             stop
%----+---------------------+-------------------------------+-------------------------------
%  1 | Sat Mar 14 10:23 PM | 2009-03-14 21:59:59.956568-06 | 2009-03-14 22:59:59.956568-06
%  2 | Sat Mar 14 09:23 PM | 2009-03-14 20:59:59.956568-06 | 2009-03-14 21:59:59.956568-06
%  3 | Sat Mar 14 08:23 PM | 2009-03-14 19:59:59.956568-06 | 2009-03-14 20:59:59.956568-06
%  4 | Sat Mar 14 07:23 PM | 2009-03-14 18:59:59.956568-06 | 2009-03-14 19:59:59.956568-06
%    \end{verbatim}
%    \normalsize
%\end{frame}
%
%\begin{frame}[fragile]
%    \frametitle{Something Fun}
%    In an different subquery, select everything from the table of the right
%    time period and right sync. Call this the "stats" query:
%    \begin{verbatim}
%SELECT
%    slavecommit,
%    EXTRACT(EPOCH FROM slavecommit - mastercommit)
%        AS synctime,
%    total
%FROM bucardo.bucardo_rate
%WHERE
%    sync = 'RO_everything' AND
%    mastercommit > (NOW() - (15 + 1) *
%        INTERVAL '1 HOUR')
%    \end{verbatim}
%\end{frame}
%
%\begin{frame}[fragile]
%    \frametitle{Something Fun}
%    ...which gives us this:
%    \footnotesize
%    \begin{verbatim}
%          slavecommit          |     synctime     | total
%-------------------------------+------------------+--------
% 2009-03-14 07:32:00.103759-06 | 5.65614098310471 |      1
% 2009-03-14 07:32:04.31508-06  | 5.25827997922897 |      3
% 2009-03-14 07:32:04.31508-06  | 5.25827997922897 |      5
% 2009-03-14 07:32:08.700184-06 | 7.71899098157883 |      1
% 2009-03-14 07:32:08.700184-06 | 8.22490698099136 |      1
% 2009-03-14 07:32:12.675518-06 | 7.85176599025726 |      6
% 2009-03-14 07:32:12.675518-06 | 7.15798497200012 |      6
%...
%    \end{verbatim}
%    \normalsize
%\end{frame}
%
%\begin{frame}[fragile]
%    \frametitle{Something Fun}
%    Now, join the two queries:
%    \begin{verbatim}
%SELECT
%    id, idname,
%    COALESCE(ROUND(AVG(synctime)::NUMERIC, 1), 0)
%        AS avgtime,
%    COALESCE(SUM(total), 0) AS count
%FROM (
%    <STATS QUERY>
%)  RIGHT JOIN (
%    <CALENDAR QUERY>
%) ON (slavecommit BETWEEN start AND stop)
%GROUP BY id, idname
%ORDER BY id DESC;
%    \end{verbatim}
%\end{frame}
%
%\begin{frame}[fragile]
%    \frametitle{Something Fun}
%    ...and get this:
%    \begin{verbatim}
% id |       idname        | avgtime | count
%----+---------------------+---------+-------
% 15 | Sat Mar 14 08:35 AM |     7.9 | 14219
% 14 | Sat Mar 14 09:35 AM |     6.9 | 16444
% 13 | Sat Mar 14 10:35 AM |     6.5 | 62100
% 12 | Sat Mar 14 11:35 AM |     6.2 | 47349
% 11 | Sat Mar 14 12:35 PM |       0 |     0
% 10 | Sat Mar 14 01:35 PM |     4.6 | 21348
%    \end{verbatim}
%    This is the average replication time and total replicated rows per hour.
%    Note that this correctly returns zeroes when no rows are replicated, and
%    still returns a value for that time slot. This prevents some amount of
%    application-side processing.
%\end{frame}
%
%\begin{frame}
%    \frametitle{Something Fun}
%    \center{That query again:}
%\end{frame}
%
%\begin{frame}[fragile]
%    \tiny
%    \begin{verbatim}
%SELECT
%    id, idname,
%    COALESCE(ROUND(AVG(synctime)::NUMERIC, 1), 0) AS avgtime,
%    COALESCE(SUM(total), 0) AS count
%FROM (
%    SELECT slavecommit,
%    EXTRACT(EPOCH FROM slavecommit - mastercommit) AS synctime,
%    total
%    FROM bucardo.bucardo_rate
%    WHERE sync = 'RO_everything' AND
%    mastercommit > (NOW() - (15 + 1) * INTERVAL '1 HOUR')
%) i
%RIGHT JOIN (
%    SELECT id, idname,
%        TO_TIMESTAMP(start - start::INTEGER % 3600) AS start,
%        TO_TIMESTAMP(stop - stop::INTEGER % 3600) AS stop
%    FROM (
%        SELECT id,
%            TO_CHAR(NOW() - id * INTERVAL '1 HOUR', 
%                'Dy Mon DD HH:MI AM') AS idname,
%            EXTRACT(EPOCH FROM NOW() - id * INTERVAL '1 HOUR') AS start,
%            EXTRACT(EPOCH FROM NOW() - (id - 1) * INTERVAL '1 HOUR') AS stop
%        FROM (
%            SELECT GENERATE_SERIES(1, 15) AS id
%        ) f
%    ) g
%) h ON (slavecommit BETWEEN start AND stop)
%GROUP BY id, idname
%ORDER BY id DESC;
%    \end{verbatim}
%\end{frame}
%
%\section{Key Points}
%\begin{frame}
%    \frametitle{Key Points}
%    \begin{itemize}
%        \item Understand join types, and use them
%        \pause
%        \item Know what functions and set operations your database provides
%        \pause
%        \item Build large queries piece by piece
%    \end{itemize}
%\end{frame}
%
%\frame{Questions?}

\end{document}

