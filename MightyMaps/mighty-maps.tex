% Mighty Maps -- Geospatial Visualization with Google Earth and KML
\documentclass{beamer}
\usetheme{Goettingen}

It seems like everyone these days has a pile of data, and the sexy thing to do
with that data is always to draw pictures of it.  <show examples of neat
graphs, and not so neat graphs> Visualization is a science, or at least an
official field of study for many people (which may or may not mean the same
thing).  People write books on how to make a good visualization, and blogs on
why everyone else's visualizations suck.

Of particular interest, since the introduction of various mapping APIs like
mapquest, and later from every search engine on the planet (Google, Bing,
Yahoo, etc.), has been geospatial visualization, for a few reasons:

* It's easy to understand. It takes a while to figure out what seemingly random
  lines and spots on a graph mean, but everyone knows what a map means
* It's easy to do, thanks to tihngs like Google Maps
* Everyone has geographic data.
    * Who browsed my web site, from where
    * Where do I ship most of my orders
    * What in fact are the migration patterns of African and European swallows?
    * These data aren't necessarily meaningful or important, but irrelevance
      rarely stops anyone from drawing graphs of their data

Having said all that, geospatial visualization continues to improve, and here
we're going to talk about one way it has done so, namely with Google Earth.
Google Earth takes Google Maps and makes it three dimensional. Users can
manipulate a 3D model of the globe, zooming in and out and examining bits of
the earth to their heart's content.  More importantly, Google Earth also lets
users add various objects to the globe using an XML-based language called
Keyhole Markup Language, or KML, after the Keyhole Corporation that developed
the technology and got bought by Google in 2004.

Here are some examples:

<blah />

So in order to give you some idea of what you can do with this stuff, let's
cover some of the elements of KML. I'll avoid showing you lots of KML syntax,
but rather just give you examples of KML objects as Google Earth displays them.
Interested parties can review the reference documentation.

Point

The basic element in most everything is a point. Points have a latitude, a longitude, an altitude, and an altitude mode
% XXX stopped at the line above
Placemark
Polygons (in limited detail)
AbstractViews
TimePrimitives
Tours -- Flyto, AnimatedUpdate
Regions

One of Earth's neater features is its ability to synchronize itself with other
instances of Earth. Linux media geeks may already be familiar with mplayer's
similar functionality, which allows you to have one mplayer instance acting as
master, and broadcasting timing information to one or more slave instances,
which is useful for things like monitor walls.
